
%%% --- From the original amsart template --- %%%

\newtheorem{theorem}{Theorem}[section]
\newtheorem{lemma}[theorem]{Lemma}

\theoremstyle{definition}
\newtheorem{definition}[theorem]{Definition}
\newtheorem{example}[theorem]{Example}
\newtheorem{xca}[theorem]{Exercise}

\theoremstyle{remark}
\newtheorem{remark}[theorem]{Remark}

\numberwithin{equation}{section}



%%% --- New environments added by us --- %%%

\theoremstyle{plain}
\newtheorem{conjecture}{Conjecture}[section]
\newtheorem{exercise}{Exercise}[section]
\newtheorem{problem}{Problem}[section]
\newtheorem{application}{Application}[section]
\newtheorem{construction}{Construction}[section]
\newtheorem{proposition}[theorem]{Proposition}
\newtheorem*{corollary}{Corollary}
\newtheorem{propdef}[theorem]{Proposition-Definition}

\theoremstyle{remark}
\newtheorem*{note}{Note}



%%% --- Packages (all added by us) --- %%%

% TODO: Do we need all these? Remove one-by-one and compile.
% Note that some of these may not be compatible with amsart; see the Author Handbook link above.

\usepackage[english]{babel}
\usepackage[utf8]{inputenc}

\usepackage{amssymb}
\usepackage{graphicx}

\usepackage[colorinlistoftodos]{todonotes}
\usepackage{hyperref}
\usepackage{tikz-cd}
\usepackage{relsize}
\usepackage[makeroom]{cancel}

\usepackage{xifthen}

% Packages for tikz
\usepackage{tikz,ulem}
\usepackage{adjustbox}
\usetikzlibrary{arrows}

%\usepackage{showkeys}

%%% --- End of packages --- %%%



%%% --- New commands added by us --- %%%

\newcommand{\N}{\mathbb{N}}
\newcommand{\Z}{\mathbb{Z}}
\newcommand{\bbP}{\mathbb{P}}
\newcommand{\PP}{\mathbb{PP}}
\newcommand{\Q}{\mathbb{Q}}
\newcommand{\C}{\mathbb{C}}
\newcommand{\Fp}{\mathbb{F}_p}
\newcommand{\Fq}{\mathbb{F}_q}

\newcommand{\defhl}[1]{\textbf{#1}}

\newcommand{\twopartdef}[4]
{
	\left\{
		\begin{array}{ll}
			#1 & \mbox{if } #2 \\
			#3 & \mbox{} #4
		\end{array}
	\right.
}

\newcommand{\threepartdef}[6]
{
	\left\{
		\begin{array}{lll}
			#1 & \mbox{if } #2 \\
			#3 & \mbox{if } #4 \\
			#5 & \mbox{} #6
		\end{array}
	\right.
}

\newcommand{\Mod}[1]{\ (\text{mod}\ #1)}

